\clearpage
%//==============================--@--==============================//%
\subsection[5.4 FSK (\textit{Frequency-Shift Keying})]{$\rightarrow$ FSK (\textit{Frequency-Shift Keying})}
\label{subsec:M-FSK}

\begin{mdframed}
    Este é um esquema de modulação não linear em que os símbolos fonte determinam a frequência de uma portadore de envelope constante. Especificamente, assume-se que o modulador consiste num conjunto de $M$ osciladores sintonizados às frequências desejadas. 

    Uma sequência de $K$ símbolos é representada pelo sinal
    $$
    v_\xi(t) = \mathbb{R}e\left\{ \sum_{k=0}^{K-1} s(t-kT) e^{j2\pi f_d \xi_k(t-kT)} e^{j2\pi f_c t} \right\},\quad 0 \leq t \leq KT
    $$
    onde as VA's discretas $\xi_k$ tomam valores no conjunto $\{2i-1-M\}_{i=1}^{M}$, e portanto, $2f_d$ é a separação entre frequências adjacentes.

    O transmissor utiliza os sinais
    \begin{align*}
        s_i(t) &= A \cos(2\pi f_i t), \qquad\qquad\quad\; 0 \leq t \leq T \\
        f_i &= f_c + (2i-1-M)f_d, \qquad i = 1,2,\dots,M 
    \end{align*}
    que têm energia comum, $E=A^2T/2$, e um envelope constante.

    \vspace{0.75em}
    \noindent Os sinais podem ser feitos ortogonais, ao escolher uma separação de frequências apropriada. Temos,
    \begin{align*}
        \int_{0}^{T} s_n(t) s_m(t)\, dt &= A^2 \int_{0}^{T} \cos(2\pi f_n t) \cos(2\pi f_m t) \, dt \\
        &= \frac{A^2}{2} \int_{0}^{T} \cos(2\pi[f_n + f_m]t)\, dt +  \frac{A^2}{2} \int_{0}^{T} \cos(2\pi[f_n - f_m]t)\, dt \\
        &= \cancel{\frac{A^2 T}{2} \frac{\sin(2\pi [f_n + f_m] T)}{2\pi [f_n + f_m] T}} + \frac{A^2 T}{2} \frac{\sin(2\pi [f_n - f_m] T)}{2\pi [f_n - f_m] T} \\
        &= \frac{A^2 T}{2} \frac{\sin(4\pi [n - m]f_d T)}{4\pi [n - m]f_d T}
    \end{align*}
    Assumimos que o produto $f_c T$ da frequência da portadora e do intervalo de símbolo é tão grande que o primeiro termo da expressão anterior pode ser descartado (\textit{bandpass assumption}). Desta forma, o produto escalar das duas formas de onda é nulo quando $4\pi f_d T$ é um multiplo não nulo de $\pi$. A menor separação que produz um par de sinais ortogonais é
    $$
        2 f_d = \frac{1}{2T}
    $$
\end{mdframed}

%//==============================--@--==============================//%
\subsubsection[5.4.1 Probabilidade de erro]{$\rightarrow$ Probabilidade de erro}

Com base na \textit{union bound}, temos o seguinte limite superior para a probabilidade de erro de símbolo
$$
    P(e) \leq \frac{M-1}{2}\, \text{erfc}\left( \sqrt{\frac{\log_2(M)}{2}\frac{E_b}{N_0}} \right)
$$

\renewcommand*{\thefootnote}{\fnsymbol{footnote}}
\footnotetext[4]{%
    A \textit{power efficiency} assintótica da modulação FSK é 
    $$
        \gamma_{\text{FSK}} = \frac{1}{2} \log_2(M)
    $$
    que, ao contrário das modulações previamente observadas, se verifica \underline{crescente} para valores de $M$ cada vez maiores.
}
\renewcommand*{\thefootnote}{\arabic{footnote}}

%//==============================--@--==============================//%
\newpage
\subsubsection[5.4.2 Espectro de potência e eficiência espectral]{$\rightarrow$ Espectro de potência e eficiência espectral}

O espectro de potência é dado por
$$
    \pmb{\dots}
$$

\vspace{0.75em}
Neste caso, $N=M$, portanto a largura de banda de Shannon deste esquema de modulação é $W=M/2T$, e a sua eficiência espectral é dada por
$$
    \left(\frac{R_s}{W}\right)_{\text{FSK}} = 2\frac{\log_2(M)}{M}
$$
Note-se que ao contrário dos esquemas anteriores (ASK, PSK e QAM), ao aumentar $M$, a eficiência espectral do FSK ortogonal \underline{diminui}. Por outro lado, a \textit{power efficiency} aumenta.

%//==============================--@--==============================//%
\subsubsection[5.4.3 Exercícios]{$\pmb{\rightarrow}$ Exercícios}
\paragraph[5.4.3.1 ]{$\pmb{\star}$ }\mbox{}\\


%//==============================--@--==============================//%