\clearpage
%//==============================--@--==============================//%
\subsection[5.3 QAM (\textit{Quadrature Amplitude Modulation})]{$\rightarrow$ QAM (\textit{Quadrature Amplitude Modulation})}
\label{subsec:M-QAM}

\begin{mdframed}
    Este é um esquema linear, em que os símbolos determinam a amplitude e a fase do sinal portadora. Ao contrário do esquema PSK, o envelope não é constante. Uma sequência $\xi$ de $K$ símbolos é representada pelo sinal
    $$
        v_{\xi}(t) = \mathbb{R}e\left\{ \sum_{k=0}^{K-1} \xi_k s(t-kT) e^{j2\pi f_0 t} \right\},\quad 0 \leq t \leq KT
    $$
    em que a VA discreta $\xi_k$ é definida como
    $$
        \xi_k \delequal \xi_k^{'} + \xi_k^{''} = A_k e^{j\phi_k} 
    $$
    e $s(t)$ é um sinal de banda base complexo de duração $T$. No caso de $s(t) = u_T(t)$, podemos reescrever a expressão como
    $$
         v_{\xi}(t) = \sum_{k=0}^{K-1} \{ \xi_k^{'} \cos(2\pi f_0 t) - \xi_k^{''} \sin(2\pi f_0 t) \} u_T(t)
    $$
    que exprime o sinal transmitido na forma de uma série de portadoras---par ortogonal---moduladas pelo conjunto de amplitudes discretas. Esta família de constelações do sinal é bi-dimensional, e o modulador e desmodulador tomam a mesma estrutura do PSK. 
\end{mdframed}

%//==============================--@--==============================//%
\subsubsection[5.3.1 Probabilidade de erro]{$\rightarrow$ Probabilidade de erro}

Um limite superior fácil para a $P(e)$ (útil para valores altos de $M$, e $E_b/N_0$ muito grandes), pode ser obtida por 
$$
    P(e) \leq 2\, \text{erfc}\left( \sqrt{\frac{3 \log_2(M)}{2(M-1)}\frac{E_b}{N_0}} \right)
$$
em que 
$$
    E = \frac{M-1}{6}\, d^2
$$

\renewcommand*{\thefootnote}{\fnsymbol{footnote}}
\footnotetext[4]{%
    A \textit{power efficiency} assintótica da modulação QAM é 
    $$
        \gamma_{\text{QAM}} = \frac{3}{2} \frac{\log_2(M)}{M-1}
    $$
    que se verifica decrescente para valores de $M$ cada vez maiores.
}
\renewcommand*{\thefootnote}{\arabic{footnote}}

%//==============================--@--==============================//%
\subsubsection[5.3.2 Espectro de potência e eficiência espectral]{$\rightarrow$ Espectro de potência e eficiência espectral}

A densidade espectral de potência do sinal QAM é dada por
$$
    S_{v_\xi}(f) = \frac{\mathbb{E}[|\xi_k|^2]}{T}\left|\tilde{S}(f)\right|^2
$$
em que $\tilde{S}(f)$ denota a Transformada de Fourier do sinal complexo $\tilde{s}(t)$.

\vspace{0.5em}
\noindent Dado que $N=2$, a largura de banda de Shannon para esta modulação é $W = 1/T$, e então, a eficiencia espectral é dada por (igual a PSK)
$$
    \left(\frac{R_s}{W}\right)_{\text{ASK}} = \log_2(M)
$$

\vspace{0.5em}
\noindent Conclui-se que para QAM, ao aumentar o valor de $M$ ocorre uma melhoria na eficiência da largura de banda, no entanto, ocorre também uma diminuição da \textit{power efficiency}.

%//==============================--@--==============================//%
\newpage
\subsubsection[5.3.3 Exercícios]{$\pmb{\rightarrow}$ Exercícios}
\paragraph[5.3.3.1 ]{$\pmb{\star}$ }\mbox{}\\



%//==============================--@--==============================//%